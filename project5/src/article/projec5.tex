\documentclass[a4paper, 10pt]{article}
\usepackage[utf8]{inputenc}
\usepackage{verbatim}
\usepackage{float}
\usepackage{listings}
\usepackage{graphicx}
\usepackage{a4wide}
\usepackage{color}
\usepackage{amsmath}
\usepackage{amssymb}
\usepackage[dvips]{epsfig}
\usepackage[toc,page]{appendix}
\usepackage[T1]{fontenc}
\usepackage{cite} % [2,3,4] --> [2--4]
\usepackage{shadow}
\usepackage{tikz}
\usepackage{hyperref}
\usepackage{titling}
\usepackage{marvosym }
\usepackage{physics}

\usepackage{subcaption}
\usepackage[noabbrev]{cleveref}
\usepackage{wasysym}
\usepackage{changepage}


\renewcommand{\topfraction}{.85}
\renewcommand{\bottomfraction}{.7}
\renewcommand{\textfraction}{.15}
\renewcommand{\floatpagefraction}{.66}
\renewcommand{\dbltopfraction}{.66}
\renewcommand{\dblfloatpagefraction}{.66}
\setcounter{topnumber}{9}
\setcounter{bottomnumber}{9}
\setcounter{totalnumber}{20}
\setcounter{dbltopnumber}{9}


\setlength{\droptitle}{-10em}   % This is your set screw

\setcounter{tocdepth}{2}

\lstset{language=c++}
\lstset{alsolanguage=[90]Fortran}
\lstset{basicstyle=\small}
\lstset{backgroundcolor=\color{white}}
\lstset{frame=single}
\lstset{stringstyle=\ttfamily}
\lstset{keywordstyle=\color{red}\bfseries}
\lstset{commentstyle=\itshape\color{blue}}
\lstset{showspaces=false}
\lstset{showstringspaces=false}
\lstset{showtabs=false}
\lstset{breaklines}
\title{FYS3150/4150 - Project 5 \\
  Disease Modeling with SIRS-model}
\author{Aram Salihi$^1$, Adam Niewelgowski$^1$ \\
  \small $^1$Department of Physics, University of Oslo, N-0316 Oslo, Norway}
\begin{document}
\maketitle
\begin{abstract}
ABSTRAKTEN HER
\end{abstract}
\tableofcontents
\newpage
\section{Introduction}
SIR (\textbf{S}usceptible, \textbf{I}nfected, \textbf{R}ecovered) model and its
variations (SIRS, MSIR, SEIR, etc.)\footnote{$https://en.wikipedia.org/wiki/Compartmental\_models\_in\_epidemiology$}
is very common method in epidemiology investigations. However this model is not only
limited to this field of studies (see e.g (REFERANSE til paperen)). The main
idea is to have three different groups, where objects are moved around between
groups based on system's conditions.
% https://www.worldscientific.com/doi/abs/10.1142/S0217984904008031
Some of the diseases are ``one-time'' diseases, i.e. one can get infected only
once. However in this project, we will allow the circular transition, it means,
when one recover, he/she can get infected again, this model is simply called
SIRS.\@

We will be using two different aproaches to study the spreadning
of a disease in the small population over time. The first approach is to
nummerically solve the coupled differential equations which are describing the
system's states and its transitions. We employ very accurate method; Runge-Kutta
of fourth order.
The second approach is the Markov Chain
Monte Carlo method, where the transition from one state/group to another is
be based on probability and random numbers.

We wish to study different variations of the model. First we look at short time
period where the population is fixed (no birth or death). We add vital dynamics
in the next variant. It means that now we allow births and deaths in the system
so the population varies. This means that the simulation needs to be run for a
longer time. We will study how some seasonal diseases impact the model by
introducing a new osciallating transmission rate. Last thing we wish to
investigate is to see how the vaccination among the population will change the
spread of the disease.

\section{Theory}
In this section we will present the theoretical framework and different types of model used in this
numerical study. Keep in mind that the algorithm Runge-Kutta is also derived in this section.

\subsection{SIR-model\label{sec:SIR}} In disease modeling the SIR model is one of the simplest
compartment models. Many complicated models are derived from this simple idea.
Consider a group of "susceptible" with population $S$, which has the possibility to be infected by a deadly or non-deadly
disease by someone which carries this infection. The person who carries this infection is in the "infected" group with population $I$.
We will now assume that there is some possibility (depends on the disease) that one or more from group $I$
heals and recoveres from the disease, and becomes immune. Thus, the persons who heals moves to the group "recovered" with population $R$.
These three variables represent the number of people in each compartment at particular time $t$. Keep in mind
that these three group are time dependent even though the number of people in each
compartment are constant. Thus the total population in this system follows
\begin{align}
  N(t) = S(t) + I(t) + R(t)
\end{align}
\subsection{SIRS-model}\label{sec:SIRS}
We will now consider a bit more general model compared with previous one presented in \eqref{sec:SIR} called SIRS.
Assuming no babies born into these three group and the possibility of dying being zero.
We will now consider a group of recovered people from a disease $\mathcal{D}$. We will now assume that these
people will have the ability to loose immunity from this disease and become susceptible. Meaning that they can
be infected by someone with disease $\mathcal{D}$.
We will now have circular system. The idea behind is demonstrated in the figure down below
\begin{center}
\begin{tikzpicture}
\def \n {3}
\def \l {S,I,R}
\def \radius {2cm}
\def \margin {6} % margin in angles, depends on the radius

\foreach \s in {1,...,\n}
{
  \draw[->, >=latex] ({360/\n * (\s - 1)+\margin}:\radius)
    arc ({360/\n * (\s - 1)+\margin}:{360/\n * (\s)-\margin}:\radius);
}
\node[draw, circle] at ({360/\n * (1 - 1)}:\radius) {$R$};
\node[draw, circle] at ({360/\n * (2 - 1)}:\radius) {$S$};
\node[draw, circle] at ({360/\n * (3 - 1)}:\radius) {$I$};
\end{tikzpicture}
\end{center}
In other words we have consider a case with a closed system. The change in the population for this system is described by following
coupled differential equations
\begin{align}
  \dv{S}{t} &= cR - \frac{aSI}{N} \\
  \dv{I}{t} &= \frac{aSI}{N} - bI \\
  \dv{R}{t} &= bI - cR
  \label{sirs}
\end{align}
Where $a$ describes the transmission rate of the disease, $b$ describes the rate of recovary, and $c$
describes the loss of immunity. The change described by these equations will always keep the
total population $N(t)$ constant. This can be easily seen when adding $S'$, $I'$ and $R'$ together.

\paragraph{Vital dynamics (open system).}\label{sec:SIRS2} We will now consider a open system. Meaning
we will have a even more realistic scenario compared with two previous models. We will now introduce
a death rate which can occour in group $S$, $I$ and $R$. The cause of death is not by the disease itself, but describing
other reasons such as natural cause of death such as heart attack. We will define this parameter rate as
$d$. Further we will introduce a death rate of people dying from the disease itself, and call this rate for $d_{i}$.
To make it even more realistic we will introduce a birth rate parameter $e$. This parameter explains the rate of babies added to the system.
We will now add a few terms to the differential equations introduced in \eqref{sirs}
\begin{align*}
  \dv{S}{t} &= cR - \frac{aSI}{N} - dS + eN\\
  \dv{I}{t} &= \frac{aSI}{N} - bI - dI - d_{I}I\\
  \dv{R}{t} &= bI - cR - dR
\end{align*}
We will further assume that all babies added to the system are susceptible.
If we take the sum of the these equation, we will notice that
\begin{align}
\dv{S}{t} + \dv{I}{t} + \dv{R}{t} = eN - dR - (d_{i} - d)I
\end{align}
From this we can see that the population number $N$ is never constant. It will either decrease or increase depending how the
factors are behaving with respect to each other.
Notice that in order to keep a increase in population with this model, $eN$ must dominate
\begin{align}
  eN >> dR + (d_{i} + d)I
\end{align}
Considering the other case, simulating an apocalypse with deadly disease/infection with a decreasing population, we must have
\begin{align}
  eN << dR + (d_{i} + d)I
\end{align}
\paragraph{Seasonal variation.}\label{sec:SIRS3} In this variant we wish to study how the
population is affected by oscillating diseases (e.g influenza). We will
slightly change the transmition rate $a$ to be time dependend
\begin{align*}
  a(t) = A\cos{(\omega t)} + a_{0}
\end{align*}
where $a_{0}$ is the average transmission rate, $A$ is the maximum deviation
from the average and $\omega$ is frequency of oscillation.

\paragraph{Vaccination.}\label{sec:SIRS4}  Here we introduce new constant, $f$,
which will break our circular model, and allow transition from \textbf{S} to
\textbf{R} directly. We want to model this constant such that this is time
depended, indicating the improvment of vaccination so more people will get
vaccinated with time.
\begin{align*}
  \frac{dS}{dt} &= cR - \frac{aSI}{N} - f\\
  \frac{dI}{dt} &= \frac{aSI}{N} - bI \\
  \frac{dR}{dt} &= bI - cR + f
\end{align*}
In section (REFERANSE til en forklaring av funksjonen til f(t)) we discuss the
choice of the function.

\subsection{Runge-Kutta 4}
Runge-Kutta of fourth order is widely used numerical algorithm for solving ordinary differential equations
,due to its precision which makes it a superior and relatively fast method. As other numerical ODE algorithms it is also based on Taylor expansion theorem.
The idea behind this method is that it provides intermediate steps to calculate
$y_{i+1}$.
Consider a general first order ordinary differential on the form:
\begin{align}
  \dv{y}{t} = f(t,y)
\end{align}
We wish to solve this ODE, i.e, we want to integrate both sides with respect to $t$.
Keep in mind that $y$ and $t$ can represent all kind of phenomenas.
\begin{align}
  y(t) &= \int f(t,y) dt
\end{align}
The idea now is to discretize our solution interval $y \to y_{i}$ and our interval $t \to t_{i}$.
We want to approximate the next solution $y_{i+1}$ by using the previous solution $y_{i}$ with a
step of
\begin{align}
  \int_{t_{i}}^{t_{i+1}}f(t,y)\dd{t}
\end{align}
Thus the equation we want to end up with
\begin{align}
  y_{i+1} = y_{i} + \int_{t_{i}}^{t_{i+1}}f(t,y)\dd{t}
\end{align}
We now wish to taylor expand our integration interval around a midpoint, thus defining
\begin{align}
  t_{i+1/2} = \frac{t_{i} + t_{i+1}}{2} = t_{i} + h/2
\end{align}








\newpage
 We first use Taylor expansion to approximate $f(t,y)$ around the midpoint of
 the integration interval, this is at $t_{i}+ h/2$, where $h$ is the step-size.
 Then we use the midpoint formula, we get
 \begin{align*}
   \int_{t_{i}}^{t_{i+1}} f(t,y) dt \approx h f(t_{i+1/2}, y_{i+1/2}) + O(h^{3})
 \end{align*}
 where we use subscript $y(t_{i}+h/2) = y_{i+1/2}$ and equivalent for time,
 $t_{i} + h/2 = t_{i+1/2}$.

 Substitution into (\ref{eq:rk4_yNext}) results in
 \begin{align*}
   y_{i+1} = y_{i} + h f(t_{i+1/2},y_{i+1/2}) + O(h^{3})
 \end{align*}
  We have to find the value of $y_{i+1/2}$ now. By employing Euler's method we
  get the approximation
  \begin{align*}
	y_{i+1/2} \approx  y_{i} + \frac{h}{2}\dd{y}{t} = y(t_{i}) + \frac{h}{2} f(t_{i}, y_{i})
  \end{align*}
  We have now derived the second-order Runge-Kutta method. Coefficients are
  given by
  \begin{align*}
	k_{1} &= h f(t_{i}, y_{i}) \\
    k_{2} &= h f(t_{i+1/2}, y_{i+1/2})
  \end{align*}
  and the update function is
  \begin{align*}
	y_{i+1} \approx y_{i} + k_{2} + O(h^{3})
  \end{align*}
  For the fourth-order Runge-Kutta method, we simply use the same idea, but
  instead of approximating the integral with midpoint rulel, we now employ the
  Simpson's rule (KANSKJE EN REFERANSE TIL DETTE, finne i boka til Morten) and
  get
  \begin{align*}
	\int_{t_{i}}^{t_{i+1}} f(t,y)dt  \approx  \frac{h}{6}\left[f(t_{i},y_{i}) +
    4f(t_{i+1/2}, y_{i+1/2}) + f(t_{i+1}, y_{i+1})\right] + O(h^{5})
  \end{align*}
  Since we do not know the values of the midpoints evaluation, we simply split
  it in two steps, so the update function becomes
  \begin{align*}
	y_{i+1} \approx y_{i} + \frac{h}{6}\left[f(t_{i},y_{i}) + 2f(t_{i+1/2},
    y_{i+1/2}) + 2f(t_{i+1/2}, y_{i+1/2}) + f(t_{i+1},y_{i+1})\right]
  \end{align*}
  Summarizing the method:
  \begin{enumerate}
  \item Find the first coefficient
    \begin{align*}
      k_{1} = hf(t_{i},y_{i})
    \end{align*}
  \item Use Euler's method to find $y_{i+1/2}$, so the next coefficient is
    \begin{align*}
      k_{2} = hf(t_{i}+h/2, y_{i}+k_{1}/2)
    \end{align*}
  \item Improve the slope calculation by finding
    \begin{align*}
      k_{3} = hf(t_{i} + h/2, y_{i}+k_{2}/2)
    \end{align*}
  \item Last coefficient is computed as follows
    \begin{align*}
      k_{4} = hf(t_{i} + h, y_{i} + k_{3})
    \end{align*}
  \item Finally, the update function is given by
    \begin{align*}
      y_{i+1} = y_{i} + \frac{1}{6}\left(k_{1} + 2k_{2} + 2k_{3} + k_{4}\right)
    \end{align*}
  \end{enumerate}



  \subsection{Monte Carlo simulation}
  Core of Monte Carlo simulation lies in randomness. Transition between
  groups/states is based on comparing a random number with a probability
  of the given transition (in-deepth description of the method can be found in
  section (REFERANSE til metode del)). However, in this section we wish only to
  show how we derive these probabilities from differential equations in section
  (\ref{sec:SIRS}).

  \vspace{3mm}

  It is explained and shown here (REFERANSE til oppgave teksten - eller skrive
  det her) how to find probabilities for the first, closed model (look
  (\ref{sec:SIRS1})). Results are following
  \begin{align*}\label{eq:SIR_trans}
	P(S \rightarrow I) = \frac{aSI}{N} \Delta t \\
    P(I \rightarrow R) = bI \Delta t \\
    P(R \rightarrow S) = cR \Delta t
  \end{align*}
  % SIRS1,2,3 og sånt dukker opp som seksjon 2.1, men når man trykker på den så
  % går det til riktig paragraf, idk om vi burde beholde det eller ikke
  Similarly we can find the transition probabilities for vital dynamics model
  (\ref{sec:SIRS2}) and vaccination model (\ref{sec:SIRS4}). Since differential
  equations differs with only some extra constants, probabilities shown in
  (\ref{eq:SIR_trans}) holds for these models as well. Respectively we get
  \begin{align*}
	P(B \rightarrow S) &= eN \Delta t\\
    P(S \rightarrow D) &= dS \Delta t\\
    P(I \rightarrow D) &= (d + d_{I})I \Delta t\\
    P(R \rightarrow D) &= dR \Delta t
  \end{align*}
  where $B$ indicates the birth group; we simply just add people to the
  \textbf{S} group. $D$ is the death group; we simply substract people from a
  given group.

  For vaccination we get
  \begin{align*}
	P(S \rightarrow R) = f \Delta t
  \end{align*}

\section{Method}
Her skal vi snakke om metode, selve implementasjon linked med teorien over.

jeg foreslår at vi har en seksjon med koden/implementation og en seksjon med
valg av parametere (og kanskje epsilon)
\section{Results}
Results are not important here. RK22 was used so everything is precise.
\section{Discussion}
Do not discuss with us coz we gon' fuck U up, bro.
\section{Conclusion}
We are the best, mUUUtherfuckers.
\section{Further work}
There is so much more we coudl do but we do not give fuck about that.

\newpage
%\bibliographystyle{apastyle}
%\bibliography{project5}
\end{document}
